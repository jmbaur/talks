\documentclass{beamer}
\usepackage{listings}
\usepackage{hyperref}
\usepackage{nix}

\usetheme{Warsaw}

\title{Nix, Nixpkgs, \& NixOS}
\author{Jared Baur}
\date{2022-03-02}

\begin{document}
\lstset{language=Nix,basicstyle=\small}

\begin{frame}
	\titlepage
\end{frame}

\begin{frame}{What is it?}
	Nix is a...
	\begin{itemize}
		\item language
		\item package manager
		\item package repository
		\item configuration tool
		\item operating system
		\item mentality
	\end{itemize}
\end{frame}

\begin{frame}{Package Manager}{ephemeral environments}
	\begin{itemize}
		\item \lstinline[language={}]!nix shell nixpkgs\#hello! drops you in an
		      interactive shell with the \lstinline!hello! package from nixpkgs
		      available
		\item \lstinline[language={}]!nix run nixpkgs\#hello! runs the
		      \lstinline!hello! package from nixpkgs and returns to your shell
		\item \lstinline[language={}]!nix run github:jmbaur/gosee! runs a
		      program I wrote that doesn't come from nixpkgs
	\end{itemize}
\end{frame}

\begin{frame}{Package Manager}{build tool}
	\begin{itemize}
		\item The base unit in Nix is a derivation
		\item A derivation can represent (almost) anything
		\item Changing a derivations inputs changes the hashed output
	\end{itemize}
	\lstinputlisting{simple-derivation.nix}
\end{frame}

\begin{frame}[containsverbatim]{Package Manager}{build tool}
	\begin{itemize}
		\item Changing a derivations inputs changes the hashed output
		\item Derivations can be addressed by their content
		\item Dependency management is no longer a problem
	\end{itemize}
	\begin{lstlisting}[language={}]
	$ ls /nix/store
	fkl40q8pjwmr5k3mz8rxvs4qczmllw6v-simple-derivation
	jpb61i1bxkfnl2ah72m8x77cjrgzksvp-simple-derivation
	...
	\end{lstlisting}
\end{frame}

\begin{frame}{Package Manager}{build tool}
	\begin{itemize}
		\item Many derivation helpers exist to enhance the build process
		      \begin{itemize}
			      \item Go: \lstinline{buildGoModule}
			      \item Python: \lstinline{buildPythonApplication}
			      \item Javascript: \lstinline{mkYarnPackage}
		      \end{itemize}
		\item Some even come with built-in linting (shellcheck will cause the
		      derivation below to fail)
	\end{itemize}
	\lstinputlisting{script.nix}
\end{frame}

\begin{frame}{development environments}
	Getting setup with a project's development environment is as easy as
	running \lstinline!nix develop!
	\lstinputlisting{dev-env.nix}
\end{frame}

\begin{frame}{Package Manager}{container images}
	\begin{itemize}
		\item Image sizes are small by default (similar to \lstinline{FROM scratch})
		\item Derivations are layered together to create an image
		\item Layers are a tree of dependencies (like all other Nix software),
		      so layer caching is smarter than Dockerfile's linear layer
		      dependency.
	\end{itemize}
	\lstinputlisting{container1.nix}
\end{frame}

\begin{frame}{Package Manager}{container images}
	\lstinputlisting{container2.nix}
\end{frame}

\begin{frame}{Package Repository}
	\url{https://github.com/nixos/nixpkgs}
	\begin{itemize}
		\item Over 65,000 packages
		\item Package upgrades/fixes/etc are just GitHub PRs
		\item Any repository conforming to the Nix flakes specification can be
		      an arbitrary repository of packages or any kind of derivation
	\end{itemize}
\end{frame}

\begin{frame}{Operating System}
	\begin{itemize}
		\item Declarative approach to systems
		\item Module system gives flexibility and abstraction
		\item Rolling back
	\end{itemize}
	\lstinputlisting{nixos-config.nix}
\end{frame}

\begin{frame}{Operating System}{testing framework}
	\lstinputlisting{nixos-test.nix}
\end{frame}

\begin{frame}{Operating System}{testing framework}
	Nix testing framework allows for integration testing entire systems
	\begin{itemize}
		\item Each node spins a separate qemu VM
		\item A node's configuration is plain NixOS configuration
		\item Tests are not limited to what you can do in a console, graphical
		      applications, screenshotting, and grabbing on-screen text are all
		      supported
	\end{itemize}
\end{frame}

\begin{frame}{Other}
	Nix can do many things, but its emphasis on reproducibility is what sets it
	apart
	\begin{itemize}
		\item Builds are ran in a sandbox that do not have access to the
		      network
		\item Network access can be obtained if the checksums of inputs are
		      known
		\item Many of the build helpers for specific languages/frameworks will
		      use locking mechanisms to aid in the fetching process
		\item Other helpers exist for calculating checksums
		      \begin{itemize}
			      \item \lstinline{nix-prefetch-url}
			      \item \lstinline{nix-prefetch-git}
			      \item \lstinline{nix-prefetch-docker}
		      \end{itemize}
	\end{itemize}
\end{frame}

\begin{frame}{Other}
	This will fail
	\lstinputlisting{fetch.nix}
\end{frame}

\begin{frame}{Other}
	This will succeed :)
	\lstinputlisting{fetch-repro.nix}
\end{frame}

\begin{frame}{Links}
	\begin{itemize}
		\item NixOS manual (\url{https://nixos.org/manual})
		\item NixOS packages \& modules (\url{https://search.nixos.org})
		\item Shipit! Presents: How Shopify Uses Nix
		      (\url{https://www.youtube.com/watch?v=KaIRpx11qrc})
		\item NYLUG Presents: Sneaking in Nix - Building Production Containers
		      with Nix (\url{https://www.youtube.com/watch?v=pfIDYQ36X0k})
	\end{itemize}
\end{frame}

\end{document}
